\documentclass[12pt]{article}

% set margins and spacing
\addtolength{\textwidth}{1.3in}
\addtolength{\oddsidemargin}{-.65in} %left margin
\addtolength{\evensidemargin}{-.65in}
\setlength{\textheight}{9in}
\setlength{\topmargin}{-.5in}
\setlength{\headheight}{0.0in}
\setlength{\footskip}{.375in}
\renewcommand{\baselinestretch}{1.0}
\linespread{1.0}

% load miscellaneous packages
\usepackage{csquotes}
\usepackage[american]{babel}
\usepackage[usenames,dvipsnames]{color}
\usepackage{graphicx,amsbsy,amssymb, amsmath, amsthm, MnSymbol,bbding,times, verbatim,bm,pifont,pdfsync,setspace,natbib}

% enable hyperlinks and table of contents
\usepackage[pdftex,
bookmarks=true,
bookmarksnumbered=false,
pdfview=fitH,
bookmarksopen=true,hyperfootnotes=false]{hyperref}

% define environments
\newtheorem{definition}{Definition}
\newtheorem{fact}{Fact}
\newtheorem{result}{Result}
\newtheorem{proposition}{Proposition}



\begin{document}
\title{Crime and Drug Treatment across Detroit}
\author{Rachel Gaudreau\thanks{Syracuse University, Economics Department. Email: regaudre@syr.edu.} \and Sophia Oritz-Heaney\thanks{Syracuse University, Economics Department. Email} \and Weston Maechling\thanks{abc} \and Leo Gershman\thanks{abc}}
\date{\vskip-.1in \today}
\maketitle

\vskip.3in
\begin{center} {\bf Abstract} \end{center}

\begin{quote}
{\small Insert abstract text here: 75-200 words, very high-level summary of your project.}
\end{quote}

\bigskip
\section{Introduction} \label{sec:introduction}

Answer the questions
\begin{enumerate}
    \item \textbf{Why should the reader care? / Why is the topic important?} (required)
    \item Why did you choose this topic? (optional)
    \item \textbf{What question will you answer? How will you do it?} (required)
        \begin{enumerate}
            \item If your theory/hypothesis fit in one paragraph, include it here. If it is longer, make it a separate section after the lit review. EITHER OPTION IS FINE as long as the length is sufficient/appropriate for your project.
        \end{enumerate}
    \item \textbf{What did you find?} (required)
    \item \textbf{Give a "road map" of the paper. Where will the reader find the various parts of your work?} (required)
\end{enumerate}

\section{Literature Review} \label{sec:literature}

Discuss at least five papers that are closely related to your results (more is better). Explain how they're related. Did you find something similar, or different? Did you look at a different context? Different time period? Different level of detail?

\section{Theoretical Analysis}
\label{sec:theory}
Optional--may include in intro if it's short.


\section{Data}
\label{sec:data}
\subsection{Overview}
\subsubsection{911 Calls in Detroit Area}
This data comes from the City of Detroit Open Data Portal's Police Serviced 911 Calls  \href{https://data.detroitmi.gov/datasets/detroitmi::police-serviced-911-calls/about}{911 Data}. It is generated by the Detroit Police Department's Crime Data Analytics when a call is placed. This data set covers 911 calls that are received by precincts around the Detroit metropolitan area from the year 2016-2022. There are a total of 1.6 million call observations that each have 26 variables. 
\textit{Make sure to add the codebook}
Variables Used: 
\begin{itemize}
    \item Longitude (per call per year)
    \item Latitude (per call per year)
    \item Year
\end{itemize}

\subsubsection{Substance Abuse Treatment Centers in Detroit Area}
This data is from the Substance Abuse and Mental Health Services Administration (SAMHSA). Data is generated through the National Survey of Substance Abuse Treatment Services \href{https://www.samhsa.gov/data/data-we-collect/n-ssats-national-survey-substance-abuse-treatment-services}{N-SSATS} an annual survey of all known public and private substance abuse treatment services in the United States. SAMHSA conducts this survey annually. This data covers the names and addresses of substance abuse treatment services in operation in the Detroit Metropolitan area between the years 2015 to 2021. There are 55 variables in total. There are 333 observations in total. There are 44 substance abuse treatment centers open in 2017.
\textit{Make sure to add the codebook}
Variables used:
\begin{itemize}
    \item Longitude (per center per year)
    \item Latitude (per center per year)
    \item Year
\end{itemize}

\subsubsection{ArcGISPro}
We used the ArcGISPro application to geocode observations from both datasets and overlap them. Using the latitude and longitude coordinates of each point, we created two new variables:
\begin{itemize}
    \item near\textunderscore\ fid: The closest drug treatment center a 911 call observation is to
    \item near \textunderscore\ dist: The distance in meters a 911 call from the closest drug treatment center
\end{itemize}
\textit{If an observation of near \textunderscore\ dist is "-1", then that observation is }

\subsection{Data Acquisition}
\subsubsection{911 Calls Data}
Use this \href{https://data.detroitmi.gov/datasets/detroitmi::police-serviced-911-calls/about}{link} to get to the City of Detroit Open Data Portal Police Serviced 911 Calls. Request data set between years of 2016 to 2022. The raw dataset should be this \href{https://www.dropbox.com/scl/fi/mvlni30fz74qx4fclofmc/calls_final.csv?rlkey=drs9rkqlgyo9i8gsf9823prof&e=1&dl=0}{file}.
\subsubsection{Substance Abuse Treatment Centers}
Use this \href{https://www.samhsa.gov/data/data-we-collect/n-ssats-national-survey-substance-abuse-treatment-services}{link} to get to SAMHSA's National Survey of Substance Abuse Treatment Services (N-SSATS). Then, request a data set for the Detroit area between 2015 and 2021. The data set should be reflective of this \href{https://github.com/ecn310/course-project-zipcentercrime/blob/main/detroit_samhsa_sud_2015_2021.dta}{file}.

\subsection{Data Manipulation}
\begin{enumerate}
    \item Download raw datasets of " "and " ". Make sure both datasets are saved as ".csv" files. (comma delimited)
    \item Open Stata (version unkown rn) and import the raw datasets. Follow this do file to isolate the observations to 2017.
    \item Upload these datasets into ArcGISPro and create XY tables for both datasets. Follow this replication file here " "
    \item Use the "near" tool in ArcGISPro to create two new variables, near\textunderscore\ fid and near \textunderscore\ dist. It will add these columns/observations to the 911 calls dataset. 
    \item Export this updated dataset as a csv file. 
    \item Open that dataset into Stata and follow this do file \href{C:\Users\sorti\OneDrive\Desktop\ArcDataDo.do.txt}{ArcData Do File}
    \item Create distance rings per distance (in meters) radius of the treatment center. The distance rings/parameters should be 50, 100, 250, 500, 750, 1000, 1250, 1500, 1750. These will become new observations in the dataset. 
    \item perform a two-sample t-test on each coinciding parameter to identify if there is a statistical difference in the distribution of observations for each. 
\end{enumerate}
Follow the master do-file/replication file here: 
\subsubsection{ArcGIS}
\subsubsection{Stata}

Describe your data. Where you got it from, how it was generated, what variables you'll use, what data cleaning steps you had to take, where your processed data, code and documentation is stored.

In a published paper, a lot of this detail will be in a data appendix. For the purposes of this report, include it all here (this may be the longest section of your report).
Variables: 
near\(_\)fid = closest treatment center to that point
near\(_\)distance= number of meters a data point is from a treatment center
\subsection{Survey data}

\section{Results}
\label{sec:result}

Explain what analyses you did, provide evidence (like in the descriptive stats exercise, but refined and clear) and then explain what your results mean.




\section{Discussion}
\label{sec:discussion}

Optional. This is where you would discuss any of the following
\begin{itemize}
    \item caveats (are there problems with the data that there are no obvious ways to resolve? if so, how might this impact
    \item future work / next steps
    \item implications of the results: that is, how your findings -- if they were causally identified -- might inform policymaking, etc.
\end{itemize}

\section{Conclusion}
\label{sec:conclusion}

Re-state (in different words) what you did and what you learned. If your discussion (Section 6) would be short, you can just have a Conclusion section that includes your discussion (that is, leave out a separate Discussion section).

\newpage
\section*{Bibliography}
\singlespacing
\setlength\bibsep{0pt}

You can either explicitly include your list of references, or you can learn to use BibTex so that it includes the references automatically.

Either way, this list should include ONLY the papers (reports, book chatpers, etc.) that you actually cite in the text (no extra).

At the same time EVERYTHING you cite in the main text must have an entry here (no references in text that don't have something here).

You can choose which citation style to follow. Whichever you choose, you must follow it consistently.

\newpage
\section*{Data Appendix} \label{sec:appendixa}
\addcontentsline{toc}{section}{Appendix A}

You should at least direct your reader to your replication package. You might put key elements of your replication package in this section as well.

\end{document}