\documentclass[12pt]{article}

% set margins and spacing
\addtolength{\textwidth}{1.3in}
\addtolength{\oddsidemargin}{-.65in} %left margin
\addtolength{\evensidemargin}{-.65in}
\setlength{\textheight}{9in}
\setlength{\topmargin}{-.5in}
\setlength{\headheight}{0.0in}
\setlength{\footskip}{.375in}
\renewcommand{\baselinestretch}{1.0}
\linespread{1.0}

% load miscellaneous packages
\usepackage{csquotes}
\usepackage[american]{babel}
\usepackage[usenames,dvipsnames]{color}
\usepackage{graphicx,amsbsy,amssymb, amsmath, amsthm, MnSymbol,bbding,times, verbatim,bm,pifont,pdfsync,setspace,natbib}

% enable hyperlinks and table of contents
\usepackage[pdftex,
bookmarks=true,
bookmarksnumbered=false,
pdfview=fitH,
bookmarksopen=true,hyperfootnotes=false]{hyperref}

% define environments
\newtheorem{definition}{Definition}
\newtheorem{fact}{Fact}
\newtheorem{result}{Result}
\newtheorem{proposition}{Proposition}



\begin{document}
\title{Crime and Drug Treatment across Detroit}
\author{Rachel Gaudreau\thanks{Syracuse University, Economics Department. Email: regaudre@syr.edu.} \and Sophia Oritz-Heaney\thanks{Syracuse University, Economics Department. Email} \and Weston Maechling\thanks{abc} \and Leo Gershman\thanks{abc}}
\date{\vskip-.1in \today}
\maketitle

\vskip.3in
\begin{center} {\bf Abstract} \end{center}

\begin{quote}
{\small Insert abstract text here: 75-200 words, very high-level summary of your project.}
\end{quote}

\bigskip
\section{Introduction} \label{sec:introduction}

Answer the questions
\begin{enumerate}
    \item \textbf{Why should the reader care? / Why is the topic important?} (required)
    \item Why did you choose this topic? (optional)
    \item \textbf{What question will you answer? How will you do it?} (required)
        \begin{enumerate}
            \item If your theory/hypothesis fit in one paragraph, include it here. If it is longer, make it a separate section after the lit review. EITHER OPTION IS FINE as long as the length is sufficient/appropriate for your project.
        \end{enumerate}
    \item \textbf{What did you find?} (required)
    \item \textbf{Give a "road map" of the paper. Where will the reader find the various parts of your work?} (required)
\end{enumerate}

\section{Literature Review} \label{sec:literature}

Discuss at least five papers that are closely related to your results (more is better). Explain how they're related. Did you find something similar, or different? Did you look at a different context? Different time period? Different level of detail?

\section{Theoretical Analysis}
\label{sec:theory}
Optional--may include in intro if it's short.


\section{Data}
\label{sec:data}
\subsection{Overview and Accquisition}

why we aren't double counting the 911 calls 
\subsubsection{911 Calls in Detroit Area (Dataset 1)}

Use this \href{https://data.detroitmi.gov/datasets/detroitmi::police-serviced-911-calls/about}{link} to get to the City of Detroit Open Data Portal Police Serviced 911 Calls. Request data set between years of 2016 to 2022. The raw dataset should be this \href{https://www.dropbox.com/scl/fi/mvlni30fz74qx4fclofmc/calls_final.csv?rlkey=drs9rkqlgyo9i8gsf9823prof&e=1&dl=0}{file}.

This data comes from the City of Detroit Open Data Portal's Police Serviced 911 Calls  \href{https://data.detroitmi.gov/datasets/detroitmi::police-serviced-911-calls/about}{911 Data}. It is generated by the Detroit Police Department's Crime Data Analytics when a call is placed. This data set covers 911 calls that are received by precincts around the Detroit metropolitan area from the year 2016-2022. We isolated the observations to one year, 2017, from the original dataset 1 so we could perform cross sectional analysis. This dataset is referred to as dataset 3. We chose observations from 2017 since the year a) overlaps with dataset 2 and b) is before COVID-19, where crime rates statistics shifted in general. (\href{https://doi.org/10.1007/s12103-023-09746-4}{Hoeboer, Kitseelar, and Henrich}). This is to eliminate COVID-19 effects from affecting our data analysis. There are a total of 1.6 million call observations that each have 26 variables. We used the longitude and latitude variables from the 2017 dataset.

\subsubsection{Substance Abuse Treatment Centers (SATC) in Detroit Area (Dataset 2)}

Use this \href{https://www.samhsa.gov/data/data-we-collect/n-ssats-national-survey-substance-abuse-treatment-services}{link} to get to SAMHSA's National Survey of Substance Abuse Treatment Services (N-SSATS). Then, request a data set for the Detroit area between 2015 and 2021. The data set should be reflective of this \href{https://github.com/ecn310/course-project-zipcentercrime/blob/main/detroit_samhsa_sud_2015_2021.dta}{file}.

This data is from the Substance Abuse and Mental Health Services Administration (SAMHSA). Data is generated through the National Survey of Substance Abuse Treatment Services \href{https://www.samhsa.gov/data/data-we-collect/n-ssats-national-survey-substance-abuse-treatment-services}{N-SSATS} an annual survey of all known public and private substance abuse treatment services in the United States. SAMHSA conducts this survey annually. This data covers the names and addresses of substance abuse treatment services in operation in the Detroit Metropolitan area between the years 2015 to 2021. We chose to use year 2017 data to be able to perform cross sectional analysis with dataset 3. There are 55 variables and 333 observations in total in dataset 2. There are 44 substance abuse treatment centers (SATC) open in 2017. We used the longitude and latitude variables from the 2017 dataset. 
\textit{Make sure to add the codebook}

\subsubsection{ArcGISPro}
We used the ArcGISPro application to layer both 2017 datasets and geocode their locations. Using the coordinates (longitude and latitude variables) of each point, we created two new variables:
\begin{itemize}
    \item near\textunderscore\ fid: This assigned every 911 call with one SATC, the center geographically closest to it. Variable is a numeric integer, with values ranging from 1 to 44. 
    \item near \textunderscore\ dist: This is the distance in meters the 911 call observation is from the SATC it is associated with in near\textunderscore\ fid. 
\end{itemize}
Since every 911 call (observation) is associated with only one SATC, no 911 call is counted twice. This is
\textit{If an observation of near \textunderscore\ dist is "-1", then that observation is 

\subsection{Data Manipulation}
\begin{enumerate}
    \item Download raw datasets of " "and " ". Make sure both datasets are saved as ".csv" files. (comma delimited)
    \item Open Stata (version unkown rn) and import the raw datasets. Follow this do file to isolate the observations to 2017.
    \item Upload these datasets into ArcGISPro and create XY tables for both datasets. Follow this replication file here " "
    \item Use the "near" tool in ArcGISPro to create two new variables, near\textunderscore\ fid and near \textunderscore\ dist. It will add these columns/observations to the 911 calls dataset. 
    \item Export this updated dataset as a csv file. 
    \item Open that dataset into Stata and follow this do file \href{C:\Users\sorti\OneDrive\Desktop\ArcDataDo.do.txt}{ArcData Do File}
    \item Create distance rings per distance (in meters) radius of the treatment center. The distance rings/parameters should be 50, 100, 250, 500, 750, 1000, 1250, 1500, 1750. These will become new observations in the dataset. 
    \item perform a two-sample t-test on each coinciding parameter to identify if there is a statistical difference in the distribution of observations for each. 
\end{enumerate}
Follow the master do-file/replication file here: 
\subsubsection{ArcGIS}
\subsubsection{Stata}

%Describe your data. Where you got it from, how it was generated, what variables you'll use, what data cleaning steps you had to take, where your processed data, code and documentation is stored.
%In a published paper, a lot of this detail will be in a data appendix. For the purposes of this report, include it all here (this may be the longest section of your report).

\section{Notes from 11/14/24:}
\begin{itemize}
    \item think about the opening and closing potentially
    \item Think about how to deal with the farther data points and potentially overlapping, sharing
    \item Only isolate 911 calls to specific drug related ones 
    \item doing zip code level with population density data and number of 911 calls in each density
    \item Only doing this in 2017 - specific we are only analyzing 2017, not just each year
    \end{itemize}
\subsection{Survey data}

\section{Results}
\label{sec:result}
%State your working hypothesis and null hypothesis - Explain why the statistics you’re presenting are appropriate -Present statistics and explain what they mean (e.g., do your results support the working hypothesis? 

Explain what analyses you did, provide evidence (like in the descriptive stats exercise, but refined and clear) and then explain what your results mean.




\section{Discussion}
\label{sec:discussion}

Optional. This is where you would discuss any of the following
\begin{itemize}
    \item caveats (are there problems with the data that there are no obvious ways to resolve? if so, how might this impact
    \item future work / next steps
    \item implications of the results: that is, how your findings -- if they were causally identified -- might inform policymaking, etc.
\end{itemize}

\section{Conclusion}
\label{sec:conclusion}

Re-state (in different words) what you did and what you learned. If your discussion (Section 6) would be short, you can just have a Conclusion section that includes your discussion (that is, leave out a separate Discussion section).

\newpage
\section*{Bibliography}
\singlespacing
\setlength\bibsep{0pt}

You can either explicitly include your list of references, or you can learn to use BibTex so that it includes the references automatically.

Either way, this list should include ONLY the papers (reports, book chatpers, etc.) that you actually cite in the text (no extra).

At the same time EVERYTHING you cite in the main text must have an entry here (no references in text that don't have something here).

You can choose which citation style to follow. Whichever you choose, you must follow it consistently.

\newpage
\section*{Data Appendix} \label{sec:appendixa}
\addcontentsline{toc}{section}{Appendix A}

You should at least direct your reader to your replication package. You might put key elements of your replication package in this section as well.

\end{document}