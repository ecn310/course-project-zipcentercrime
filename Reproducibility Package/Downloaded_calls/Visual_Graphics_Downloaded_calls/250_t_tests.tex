\begin{table}[htbp]
\centering
\begin{tabular}{l|c c c c}
\hline
Comparison & Mean 1 & Mean 2 & Difference & P-value \\
\hline
250m-500m & 1086.6 & 626.0 & 460.5 & 0.000 \\
500m-750m & 626.0 & 517.2 & 108.8 & 0.071 \\
750m-1000m & 517.2 & 392.1 & 125.1 & 0.019 \\
1000m-1250m & 392.1 & 337.7 & 54.4 & 0.011 \\
1250m-1500m & 337.7 & 211.1 & 126.7 & 0.000 \\
1500m-1750m & 211.1 & 165.8 & 45.3 & 0.002 \\
1750m-2000m & 165.8 & 123.2 & 42.5 & 0.005 \\
2000m-2250m & 123.2 & 98.2 & 25.0 & 0.007 \\
\hline
\end{tabular}
\caption{\textbf{One-sample T-test Results by 250m Distance Rings}}
\label{tab:ttests1}
\centering\textit{This table shows one sample t-test in the difference of mean all calls per SATC per distance group. We got the means for each group by first assigning every 911 call the distance bucket they fall in respective to the SATC they are geographically closest to. Afterward, we divide each ring by its concentric ring's area* 1,000,000. We multiplied by 1,000,000 to offset the amount of negative number spaces seen when we divide total calls by their large areas. This way, our total and mean calls per area are not in the negatives so they are easier to see in the table. }
\end{table}
