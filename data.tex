\documentclass[12pt]{article}

% set margins and spacing
\addtolength{\textwidth}{1.3in}
\addtolength{\oddsidemargin}{-.65in} %left margin
\addtolength{\evensidemargin}{-.65in}
\setlength{\textheight}{9in}
\setlength{\topmargin}{-.5in}
\setlength{\headheight}{0.0in}
\setlength{\footskip}{.375in}
\renewcommand{\baselinestretch}{1.0}
\linespread{1.0}

% load miscellaneous packages
\usepackage{csquotes}
\usepackage[american]{babel}
\usepackage[usenames,dvipsnames]{color}
\usepackage{graphicx,amsbsy,amssymb, amsmath, amsthm, MnSymbol,bbding,times, verbatim,bm,pifont,pdfsync,setspace,natbib}

% enable hyperlinks and table of contents
\usepackage[pdftex,
bookmarks=true,
bookmarksnumbered=false,
pdfview=fitH,
bookmarksopen=true,hyperfootnotes=false]{hyperref}



\begin{document}
\title{Insert group name here}
% add a fourth name if you have four team members; fill in at least first names below
\author{Name 1\thanks{Syracuse University, Economics Department. Email: } \and Name 2\thanks{abc} \and Name 3\thanks{abc}}
\date{\vskip-.1in \today}
\maketitle

\vskip.3in

\section{Research Question} \label{sec:question}

State the current version of your research question.

\section{Data Overview} \label{sec:literature}

Write a short description of each  data set. Use subsections as demonstrated below if you have more than one data set.

\subsection{Data Set 1 (insert a descriptive title)}
\begin{itemize}
  \item What is the source of the data?
  \item How was it generated? Some possibilities: Did someone do a survey? Do people submit their own information? Did an organization compile it from public reports? 
  \item What is the coverage of the data?
    \begin{itemize}
        \item What years?
        \item What kind of observations (countries? individual survey respondents? etc)
        \item Scope (e.g., if countries, only developed countries?)
    \end{itemize}
  \item Number of variables
    \begin{itemize}
        \item Types of variables,  (is this dataset giving mainly economic indicators for a country? Is it reporting details about 911 calls?
    \end{itemize}
  \item How many observations total? (e.g., 20 countries with 10 variables each, for a total of 100 observations)
\end{itemize}

\subsection{Data Set 2 (insert a descriptive title)}

Answer all the same questions as in the previous subsection. \\

\noindent Add additional subsections if you have more than two data sets.


\section{Data Acquisition}
\label{sec:theory}

Explain how someone can acquire each dataset (you may need to use subsections again).
\begin{itemize}
    \item This should be in EXCRUCIATING detail: every click someone has to execute to get the dataset as you have it
\end{itemize}

\noindent Where have YOU stored the data? Include link(s).
\begin{itemize}
    \item Example of how to make a url into an explicit link: \url{https://github.com/ecn310/course-project-zipcentercrime/tree/main/ArcGIS%20files}
    \item Example of how to make a piece of text into a hyperlink: \href{https://github.com/ecn310/course-project-zipcentercrime/tree/main/ArcGIS%20files}{ArcGIS files}
    \item If your code and documentation is in a different location, where is it?
\end{itemize}


\section{Data Manipulation}
\label{sec:data}

What steps have you taken / do you plan to take to get the data ready for analysis? 
\begin{itemize}
    \item Examples: checking for outliers; dropping some observations you don't need; investigating something you've already noticed that doesn't make sense.
    \item If you've already done some work, where are the files and documentation for that work?
\end{itemize}

\section{Linking Datasets}
\label{sec:discussion}

How you will link different data sets together (if you'll need to)
\begin{itemize}
    \item What variable(s) in each dataset will you use to merge the datasets together?
    \item Example: We will merge Dataset A with GDP for each country with Dataset B that has manufacturing employment for each country across years. In Dataset A, these variables are called \textit{year} and \emph{country}; Dataset B calls the year variable \emph{yr} and the country variable \emph{name}.
\end{itemize}


\section{Key Variables}
\label{sec:result}

List of key variables you may use for testing your hypothesis, separately for each dataset





\end{document}